\chapter{Model}
The initial goal of this project was to look into methods to not use ruler presence in the
diagnosis classification.
To be able to do this, a model that was using the ruler presence in its predictions was needed.
Preferably this model should also perform close to the best performing model, to be comparable.

\section{ResNet18 architecture trained with MixUp}
On the Kaggle classification competetion related to the HAM10000 dataset\cite{HAM10000-kaggle-competetion},
another user claimed to get a $97.7\%$ accuracy using a ResNet18 model trained with MixUp\cite{kaggle-97-model}.
When the provided model was retrained, the results were not as good as claimed.
The actual accuracy was in the range of $90\%$ to $92\%$, depending on the data split.
That is however still close to the best performing model published (see \ref{sec:state-of-the-art}).

The actual implementation of the model can be seen in Appendix \ref{appendix:resnet-18-mixup}.
Using the slight modifications to the original mode, the model reached an accuracy of roughly $91\%$.

\subsection{Model analysis}
\subsubsection{Prediction saliency map}
As described in the Introduction section, there is an academically described risk,
that the model will use the presence of rulers in its predictions. % Maybe a reference here?

We will first test this, by creating saliency maps (\ref{sec:saliency_maps}) for some of the images with rulers in the dataset.
These can be seen in Figure \ref{fig:ruler_saliency_map}.
More similar examples can be seen in Appendix \ref{appendix:ruler_saliency_maps}.

\begin{figure}[h]
    \includegraphics[
        width=\textwidth,
        height=\textheight,
        keepaspectratio=true,
        angle=0,
        clip=false
    ]{build/saliency_maps/overview_map_2.png}
    \caption{Saliency maps of the model prediction on an image with a ruler.}
    \label{fig:ruler_saliency_map}
\end{figure}

\subsubsection{Prediction precision on different classes}
To further evaluate the model, we will investigate wether it underperforms on some classes.
In Figure \ref{fig:prediction_strength} a confusion matrix for the trained model is shown.

\begin{figure}[ht]
    \centering
    \includegraphics[
        width=0.7\textwidth,
    ]{build/prediction_strength/confusion_matrix_seaborn.png}
    \caption{Confusion matrix of the model prediction on the dataset. 
        Normalization has been done over the truth.
    }
    \label{fig:prediction_strength}
\end{figure}

The confusion matrix shows that the model underperforms on the \verb|mel| (melanoma) class.
For terms of the using this model in practice, this would be problematic and should be adressed.
The model does however perform fairly well in general, so it will still be used to investigate the
impact of confounding elements on its predictions.

\subsubsection{Prediction strength on the melanoma class}
Assuming that the rulers do indeed affect the models predictions,
it would likely impact the prediction strength of the model if rulers are present. 
Since the rulers are overly present in the images of lesions with melanoma,
it would under the assumption be expected, that the presence of the rulers improves the model's predictions 
on the \verb|mel| class.

To test this, a plot has been created below that shows the prediction strength of the model on the \verb|mel| class,
seperated over the presence of rulers (Figure \ref{fig:prediction_strength_mel}).

It shows a slightly better melanoma prediction precision on the pictures with rulers (Figure \ref{fig:prediction_strength_mel_normalized}).
Doing a $\chi^2$ test on the data from Figure \ref{fig:prediction_strength_mel_not_normalized},
however shows that the difference is not significant ($p=\input{build/prediction_strength/p_mel.txt}).


\begin{figure}
    \centering
    \begin{subfigure}[h]{0.45\textwidth}
        \includegraphics[
            width=\textwidth,
        ]{
            build/prediction_strength/mel_confusion_matrix_seaborn.png
        }
        \caption{No normalization}
        \label{fig:prediction_strength_mel_not_normalized}
    \end{subfigure}
    \begin{subfigure}[h]{0.45\textwidth}
        \includegraphics[
            width=\textwidth,
        ]{
            build/prediction_strength/mel_confusion_matrix_seaborn_normalized.png
        }
        \caption{Normalized over the presence of rulers}
        \label{fig:prediction_strength_mel_normalized}
    \end{subfigure}
    \caption{Confusion matrix of the model prediction the melanoma cases split up by presence of a ruler.}
    \label{fig:prediction_strength_mel}
\end{figure}


\end{figure}