\section{Debiasing Skin Lesion Datasets and Model. Not So Fast}
In their paper
''Debiasing Skin Lesion Datasets and Model. Not So Fast''
Bissoto, Alceu and Valle look into debiasing models trained on
skin lesion images \cite{debias-not-so-fast}.
They go through $7$ differenct artifacts in the images that could
be confunding elements, and try state of the art methods to
make them disregard the artifacts.
These include the rulers, that is the focus of this report.
The paper makes several claims about biases that occur in
skin lesion models.
Their focus is the removal of bias, but in that proces they
make some claims about the model actually being biased.
Two of these will be examined here.

\textbf{Claim 1:}
The model architecture is capable of detecting the artifacts in the images.

\textbf{Claim 2:}
A model trained to detect malignant lesions, is basing its predictions
partly on the detected artifacts.

\subsection{Claim 1: Capability of the model architecture}
To show that the model architecture used for lesion clasification is
indeed capable of detecting the artifacts in the images, they authors
label the images with the artifacts and train a model to identify
each of them.

The ruler classification model reaches an implessive AUC of $98.2\%$
on the ISIC dataset (see Figure 4 of the paper\cite{debias-not-so-fast}).

From this, they conclude that the model architecture is capable of identiying
the rulers (and similar artifacts) in the images.

\subsection{Claim 2: The lesion classification model is biased}


