\chapter{Testing the hypthesis}
From both the background litterature and general worry about models,
there is reason to believe that a model trained on the HAM10000 dataset will be biased
towards looking at confounding elements - hereunder the rulers in the image.
In the following tis hypothesis will be tested using both methods that other researchers have used,
and also other methods checking the output of the model.

\section{Prediction saliency map}\label{sec:prediction-saliency-map}
In \textit{Interpretations are useful: Penalizing Explanations to Align Neural Networks with Prior Knowledge}\cite{interps-are-useful},
(described in Section \ref{sec:interps-are-useful}), it is shown that a model trained on skin lesion images,
will mark out the ruler on malignant images containing one.
With the RESNET model described in \ref{sec:model}, similar saliency maps are constructed.

The saliency maps made here are gradient based as described in Section \ref{sec:gradiant_saliency_maps}.
In Figure \ref{fig:ruler_saliency_map} these can be seen.
More similar examples can be seen in Appendix \ref{appendix:ruler_saliency_maps}.

\begin{figure}[h]
    \includegraphics[
        width=\textwidth,
        height=\textheight,
        keepaspectratio=true,
        angle=0,
        clip=false
    ]{build/saliency_maps/overview_map_2.png}
    \caption{Saliency maps of the model prediction on an image with a ruler.}
    \label{fig:ruler_saliency_map}
\end{figure}

These images don't show results nearly as clearly as the ones from the article.
Where the paper had results showing the ruler very clearly 0 these images will often just show a half of it.

\section{Prediction precision on different classes}
To further evaluate the model, we will investigate wether it underperforms on some classes.
In Figure \ref{fig:prediction_strength} a confusion matrix for the trained model is shown.

\begin{figure}[ht]
    \centering
    \includegraphics[
        width=0.7\textwidth,
    ]{build/prediction_strength/confusion_matrix_seaborn.png}
    \caption{Confusion matrix of the model prediction on the dataset. 
        Normalization has been done over the truth.
    }
    \label{fig:prediction_strength}
\end{figure}

The confusion matrix shows that the model underperforms on the \verb|mel| (melanoma) class.
For terms of the using this model in practice, this would be problematic and should be adressed.
The model does however perform fairly well in general, so it will still be used to investigate the
impact of confounding elements on its predictions.

\subsection{Feature based nearest neighbors}
In \textit{Debiasing Skin Lesions Datasets and Model. Not so Fast}\cite{debias-not-so-fast} (described in Section \ref{sec:debias-not-so-fast}),
the authors argue that their melanoma prediction can pick up on confounding elements (like rulers), 
by examining the internal layers of their model and comparing them to one another.
Specifically, they find, that the vectors outputted from internal layers,
that are thought of as representing the semantic features of an image,
have a short euclidean distance to one another.
For a detailed description see section \ref{sec:debias-not-so-fast}.

To check if this is the case for our model, we use the same approach and make a similar plot as seen in Figure \ref{fig:not-so-fast-artifact-query}.

\subsection{Prediction strength on the melanoma class}
Assuming that the rulers do indeed affect the models predictions,
it would likely impact the prediction strength of the model if rulers are present. 
Since the rulers are overly present in the images of lesions with melanoma,
it would under the assumption be expected, that the presence of the rulers improves the model's predictions 
on the \verb|mel| class.

To test this, a plot has been created below that shows the prediction strength of the model on the \verb|mel| class,
seperated over the presence of rulers (Figure \ref{fig:prediction_strength_mel}).

It shows a slightly better melanoma prediction precision on the pictures with rulers (Figure \ref{fig:prediction_strength_mel_normalized}).
Doing a $\chi^2$ test on the data from Figure \ref{fig:prediction_strength_mel_not_normalized},
however shows that the difference is not significant ($p=\input{build/prediction_strength/p_mel.txt}).


\begin{figure}
    \centering
    \begin{subfigure}[h]{0.45\textwidth}
        \includegraphics[
            width=\textwidth,
        ]{
            build/prediction_strength/mel_confusion_matrix_seaborn.png
        }
        \caption{No normalization}
        \label{fig:prediction_strength_mel_not_normalized}
    \end{subfigure}
    \begin{subfigure}[h]{0.45\textwidth}
        \includegraphics[
            width=\textwidth,
        ]{
            build/prediction_strength/mel_confusion_matrix_seaborn_normalized.png
        }
        \caption{Normalized over the presence of rulers}
        \label{fig:prediction_strength_mel_normalized}
    \end{subfigure}
    \caption{Confusion matrix of the model prediction the melanoma cases split up by presence of a ruler.}
    \label{fig:prediction_strength_mel}
\end{figure}


\end{figure}