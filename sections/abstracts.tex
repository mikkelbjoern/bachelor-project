\begin{center}
\vspace*{\fill}
\begin{large}
    \textbf{Preface and Acknowledgements}
\end{large}

This project is written as my thesis for my bachelor's degree in Mathemathics and Technology at
the Danish Technical University.
The code for generating this report and all figures within it can be found on 
\href{https://github.com/mikkelbjoern/bachelor-project}{GitHub}
\github{https://github.com/mikkelbjoern/bachelor-project}.

It has been written at the Department of Applied Mathematics and Computer Science and supervised by
Professor Aasa Feragen and PhD-student Manxi Lin. 

I would like to acknowledge and give my warmest thanks both of them for all of their help and support.
Their quick and insightful feedback along the way got me through the research process and the writing of this report.




\end{center}
\vspace*{\fill}
\pagebreak
\vspace*{\fill}
\selectlanguage{english} 
\begin{abstract}
Deep learning models can be used to classify medical images.
This may be problematic, as models reportedly use confounding elements (elements
 in the images that are not supposed to be used in the classification),
in their predictions.
Multiple studies have been conducted to understand, demonstrate and
mitigate the effects of these elements.

This project investigates the impact of one example,
namely, rulers in images of skin lesions originally used by 
doctors to monitor lesion growth.
A close to state-of-the-art classification model is trained, and
using methods of other researchers as well as statistical tests on the model output,
it is concluded, that the model is not using these rulers.
Further, it is argued that other models trained on the same data,
are also unlikely to use the rulers in their predictions.
It is also seen that in general models trained with access to less
information around the lesions perform better, suggesting that
the confounding elements contained there are not necessary
for the models predictions.
Finally, it is recommended that
care is taken when making claims about the impact of a confounding 
element on a given model.

\end{abstract}
\vspace*{\fill}
\selectlanguage{danish} 
\begin{abstract}
Modeller baseret på dyb læring kan bruges til at klassificere medicinske billeder.
Det er muligvis problematisk, da forskningsresultater viser at sådanne modeller 
kan gøre brug for elementer på billederne der ikke ønskes brugt i klassificeringen.

Dette projekt undersøger effekten af et sådant element, 
nemlig linealer på billeder af hudtumorer oprindeligt brugt af læger til at holde 
øje med hvor hurtigt tumoren vokser.
Der trænes en model der klarer sig tæt på de bedste lavet på lignende datasæt,
og ved at teste den med metoder brugt af andre forskere samt statistiske tests på
modellens forudsigelser, konkluderes det at modellen ikke gør brug af linealerne.
Det vises også at modeller der trænes med mindre data omkring billedeinformation 
omkring tumoren klarer sig bedre, hvilket tyder på, at modellen ike bruger de 
omkringliggende uønskede elementer i sin klassificering. 
Til sidst anbefales det at man er forsigtig med at konkludere at en model bruger
elementer den ikke bør.

\end{abstract}
\selectlanguage{english}
\vspace*{\fill}
