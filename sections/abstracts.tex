\begin{center}
\vspace*{\fill}
\begin{large}
    \textbf{Preface and Acknowledgements}
\end{large}

This project is written as my thesis for my bachelor's degree in Mathemathics and Technology at
the Danish Technical University.
It has been written at the Department of Applied Mathematics and Computer Science and supervised by
Professor Aasa Feragen and PhD-student Manxi Lin. 

I would like to acknowledge and give my warmest thanks both Aasa and Manxi for all of their help and support.
Their quick and insightful responses got me through the research process and the writing of this report.



\end{center}
\vspace*{\fill}
\pagebreak
\vspace*{\fill}
\selectlanguage{english} 
\begin{abstract}
Deep learning models used to classify skin lesions are reported
to be using so-called confounding elements in its predictions.
This behavior is undesirable for multiple reasons. 
This project investigates the impact of one example,
namely, rulers in images of skin lesions originally used by 
doctors to monitor lesion growth.
It is tested in multiple ways if the model is using the rulers in
its predictions.
It is concluded, that the model is not using these rulers,
and that it is unlikely that other models trained on the same data
are using them.
It is also seen that in general models trained with access to less
information around the lesions perform better, suggesting that
the possibly confounding elements contained there are not necessary
for the models predictions.

\end{abstract}
\vspace*{\fill}
\selectlanguage{danish} 
\begin{abstract}
Abstract på dansk
\end{abstract}
\selectlanguage{english}
\vspace*{\fill}
