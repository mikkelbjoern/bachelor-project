\chapter{Theory}
This report assumes that the reader is familiar with the basics of machine learning.
This includes basic linear algebra, data preprocessing and model selection.
Further a basic understanding of how neural networks function is assumed,
including the basic concept of gradiant descent used for model training.


\section{Tensors}
The following definitions and concepts are heavily based on \textit{Introduction to Convolutional Neural Networks} by J. Wu \cite{tensorIntroduction}.
In linear algebra, the concept of a vector is introduced.
Further a 2-dimensional vector is introduced, the \textit{matrix}.
We will denote a vector as $\vecsym{v}\in \mathbb{F}^n$ and a matrix as $\matsym{M}\in \mathbb{F}^{m\times n}$.
Here $\mathbb{F}$ is a field.
For the purpose of this report, a field can be considered a set of element with a defined addition and multiplication operation.
Especially most of the time the field will be the real numbers, $\mathbb{R}$.

Having defined vectors and matrices, we can now define a tensor as the natural extension of the matrix abstraction,
by adding more dimensions.
A $3$-dimensional tensor is can then be defined as $\tenssym{T}\in \mathbb{F}^{m\times n\times c}$.
Such a tensor is a natural way to represent an image in colors.
Such a representation of an image of height $h$ and width $w$ could then be represented as an
element in $\mathbb{F}^{h\times w\times 3}$.

An example of a higher order tensor used in this project, could be a batch of images.
Such a batch could be represented as a tensor of shape $(h, w, c, b)$ where $b$ is the batch size. 
That is, the entire batch would be a $4$-dimensional tensor in the tensor-space $\mathbb{F}^{h\times w\times c\times b}$.

\subsection{Vectorizing tensors}
As noted in most introductions of linear algebra, matrices are themselves vectors.
This is to be understood as, any matrix $\matsym{M} \in \mathbb{F}^{m\times n}$ can be represented as a vector $\vecsym{m} \in \mathbb{F}^{m\cdot n}$.
This is commonly done by stacing the columns of the matrix, e.g.  if 
\begin{equation}
\matsym{M} = \begin{bmatrix}
    m_1 & m_2 & m_3 \\
    m_4 & m_5 & m_6 \\
    m_7 & m_8 & m_9
\end{bmatrix} \in \mathbb{F}^{3 \times 3}
\end{equation}
then the vector representation is
\begin{equation}
\vecsym{m} = \begin{bmatrix}
    m_1 & m_4 & m_7 & m_2 & m_5 & m_8 & m_3 & m_6 & m_9
\end{bmatrix}^T \in \mathbb{F}^{9}
\end{equation}
The same concept can be applied to tensors of higher dimensions than the $2$-dimensional matrix,
by stacking the columns continously.

\subsection{Partial derivatives over vectors} \label{sec:partial_derivatives_of_scalar_over_vector}
In multiple setings in training of neural networks, it is neccesarry to figure out what
parts of the model contributed to the result of the model.
During training, we for instance want to know what parts of the model contributed to an increase of the loss function,
so that these parts can be changed thereby hopefully improving the model.
In this report it is also used for creating saliency maps, and discussing them (see section \ref{sec:gradiant_saliency_maps}).

Defining the partial derivative of a vector $\vecsym{v}\in\mathbb{F}^n$ with respect to a scalar $s$ is done as follows:
\begin{equation}
    \left[ \frac{\partial s}{\partial \vecsym{v}} \right]_i 
         = \frac{\partial s}{\partial \vecsym{v}_i}
\end{equation}

Note that with this definition, derivatives over matrices can also be done by first vectorizing them,
and we then get the nice algebraic property that  $\frac{\partial s}{\partial \vecsym{v}}^{T} = \frac{\partial s}{\partial \vecsym{v}^{T}}$ 


\subsection{Partially deriving vector functions}
Suppose $\vecsym{x} \in \mathbb{F}^n$, 
$f: \mathbb{F}^n \rightarrow \mathbb{F}^m$ and
$\vecsym{y} = f(\vecsym{x}) \in \mathbb{F}^m$.
Then
\begin{equation}
    \left[ \frac{\partial \vecsym{y}}{\partial \vecsym{x}^T} \right]_{ij} =
    \frac{\partial \vecsym{y}_i}{\partial \vecsym{x}_j}
\end{equation}

Which means that the derivative of a vector function on the form of $f$ will be a matrix/$2$-tensor
in the tensor-space $\mathbb{F}^{m\times n}$.


\section{Convolutional neural networks (CCNs)} \label{sec:convolutional_neural_networks}
Convolutional neural networks (CNNs) are a fairly modern method used in image classification.
They are based on the idea of a convolutional layer, which is a way of combining pixels in a
neighbourhood of an image into single features.

The mathemathical basis is the \textit{convolution} of a matrix.
During the definition of convolutions, the following matrix will be used as an example:
\begin{equation}
    \matsym{I} = \input{build/convolution_example/I.tex}
\end{equation}

Think of $\matsym{I}$ as an image.
Sematically there is an area in the \"image\" that has way higher values than the rest
(the lower right corner).
A convolution on the image, will reduce this information to a more compact representation.
To do this, a matrix kernel needs to be defined. It's standard to use a square kernel.
Here we will use a $3 \times 3$ kernel:
\begin{equation}
    \matsym{K} = 
        \input{build/convolution_example/K.tex}
\end{equation}

To then convolve $\matsym{I}$ with $\matsym{K}$, we \textit{move} the kernel over the image,
and multiply each index of the kernel with the index that it is currently \textit{above} in the image
(a full mathemathical deifintion follows lates in section \ref{sec:convolution_definition}).
That will in this case result in the matrix:
\begin{equation}
    \input{build/convolution_example/convolution.tex}
\end{equation}


On Figure \ref{fig:convolution_example} heatmaps of the original $I$ and the convolved, shows that they look very similar.
\begin{figure}[h]
\centering
\begin{subfigure}{0.45\textwidth}
    \includegraphics[width=\textwidth]{build/convolution_example/I.png}
    \caption{Original version of $I$}
\end{subfigure}
\begin{subfigure}{0.45\textwidth}
    \includegraphics[width=\textwidth]{build/convolution_example/convolution.png}
    \caption{Convolution of $I$ with $K$}
\end{subfigure}
\caption{Heatmaps of the original $I$ and the convolved $I$}
\label{fig:convolution_example}
\end{figure}

Comparing the heatmaps on Figure \ref{fig:convolution_example}, makes it clear that little to no imformation about
where the high values are in the picture has been lost.
The amount of data points has however been reduced.

\subsection{Padding}
The current model of moving the kernel over the image, will not prioritize the borders
of the image a lot. 
To compat this, the image is often padded - usually with zeros. 
This is done by adding a border of zeros around the image, the width of which is refered to as the size of the padding.
By default, the padding in libaries like PyTorch \cite{PyTorch} is $0$ (no padding) as in the example before.

To continue the example, we will now use a padding of $1$ around the image.
\begin{equation}
    \matsym{I}_{\text{padded}} = \input{build/convolution_example/I_padded.tex}
\end{equation}

Calculating the convolution of the padded image with the kernel, results in the following matrix:
\begin{equation}
    \input{build/convolution_example/convolution_padded.tex}
\end{equation}

And again plotting the heatmaps we see two similar images again:
\begin{figure}[h]
\centering
\begin{subfigure}{0.45\textwidth}
    \includegraphics[width=\textwidth]{build/convolution_example/I.png}
    \caption{The original $\matsym{I}$}
\end{subfigure}
\begin{subfigure}{0.45\textwidth}
    \includegraphics[width=\textwidth]{build/convolution_example/convolution_padded.png}
    \caption{Convolution of $\matsym{I}_{\text{padded}}$ with $K$}
\end{subfigure}
\caption{Heatmaps of the padded $\matsym{I}_{\text{padded}}$ and the convolved $\matsym{I}_{\text{padded}}$}
\label{fig:convolution_example_padded}
\end{figure}

Note that padding increases the amount of data points in the image, but that
the effect of this is not as extreme as it seems in the example, since images a usually
a lot larger than this small example.

\subsection{Stride}
To reduce the amount of data points even further, we can use a stride.
This is done by moving the kernel over the image as before, but moving multiple pixelse each time.

Following the example from before using a stride of $2$ and a padding of $1$, the convolution results in the following matrix:
\begin{equation}
    \input{build/convolution_example/convolution_stride.tex}
\end{equation}

This is again plotted in Figure \ref{fig:convolution_example_stride}.

\begin{figure}[ht]
\centering
\begin{subfigure}{0.45\textwidth}
    \includegraphics[width=\textwidth]{build/convolution_example/I.png}
    \caption{The original $\matsym{I}$}
\end{subfigure}
\begin{subfigure}{0.45\textwidth}
    \includegraphics[width=\textwidth]{build/convolution_example/convolution_stride.png}
    \caption{Convolution of $\matsym{I}_{\text{padded}}$ with stride $2$}
\end{subfigure}
\caption{Heatmaps of the original $\matsym{I}$ and the convolved $\matsym{I}_{\text{padded}}$}
\label{fig:convolution_example_stride}
\end{figure}

It is again clear that the data seems similar to the original image, but the data points
has been reduced from $\input{build/convolution_example/I_size.tex}$ to $\input{build/convolution_example/final_convolution_size.tex}$.

In convolutions can combine information from data where some kind of \textit{proximity} is important.
This is very much the case in image analysis (hence the focus in this explanation), but the idea is
not limited to this field.

\subsubsection{Full definition of convolutions}\label{sec:convolution_definition}
The above is a very simple example of a convolution.
A full compact formula for making a convolution is left out, as it is not readable and
ultimately not very insightful. 
Instead we will make a implementation of the a \verb|convolve| function in Python.
To define such an algorithm, it is usefull to know the output size of the final matrix.
Let the width of the original image be $w$, the kernel size be $k$, the padding be $p$, and the stride be $s$.
Calculating the output width can then be done by finding all the possible values of $w$ that can be
used as the left value of the kernel.
First the padded size of the image is seen to be $w + 2p$.
Since a kernel is moved over the picture, the last $k-1$ pixels cant be used to place a kernel.
Assuming that the image is non-empty and the kernel is smaller than or the same size as the image,
at least one kernel can be placed, adding up to a total of $k$ pixels where a kernel cannot be placed.
Of the remaining only $1/s$ of the pixels are actually used to place a kernel, hence the output width becomes:
\begin{equation}
    w_{out} = \left\lfloor \frac{w + 2p - k}{s}\right\rfloor + 1
\end{equation}

And with a completely similar procedure for the height, $h$, we get:
\begin{equation}
    h_{out} = \left\lfloor \frac{h + 2p - k}{s} \right\rfloor + 1
\end{equation}

Having these values, we can then define a Python function that performs the convolution:

\inputminted[]{python}{src/convolve.py}

\section{The power of convolution kernels}
In the example in Section \ref{sec:convolutional_neural_networks}, a convolution that
comprimed the image into a condensed version was used.

Kernels can do lot more than compressing images though.
For the following, different types of kernels are used to convolute an image of a Model of the Greek Temple of Artemis \cite{greek-temple-picture}.

With a kernel like this:
\begin{equation}
    K_{\text{edge}} = \input{build/kernel_examples/K_edge.txt} 
\end{equation}
an image is produced where the edges are highlighted.

\begin{figure}[ht]
\centering
\begin{subfigure}{0.45\textwidth}
    \includegraphics[width=\textwidth]{build/kernel_examples/image_scaled.jpg}
    \caption{The original image}
\end{subfigure}
\begin{subfigure}{0.45\textwidth}
    \includegraphics[width=\textwidth]{build/kernel_examples/edges.jpg}
    \caption{The image considered as a matrix convolved with $K_{\text{edge}}$}
\end{subfigure}
\caption{The image with the edges highlighted}
\end{figure}

In a similar way, a kernel like this:
\begin{equation}
    K_{\text{vertical}} = \input{build/kernel_examples/K_vertical.txt}
\end{equation}
can be used to highlight the vertical lines in the image.

\begin{figure}[ht]
\centering
\begin{subfigure}{0.45\textwidth}
    \includegraphics[width=\textwidth]{build/kernel_examples/image_scaled.jpg}
    \caption{The original image}
\end{subfigure}
\begin{subfigure}{0.45\textwidth}
    \includegraphics[width=\textwidth]{build/kernel_examples/vertical.jpg}
    \caption{The image considered as a matrix convolved with $K_{\text{vertical}}$}
\end{subfigure}
\caption{The image with the vertical lines highlighted}
\end{figure}

A lot of other interesting kernels exist, see for instance these examples at AI Shack \cite{kernel-example-webpage}.

The point we want to make here, is that a lot of different information can be extracted from an image,
if the correct kernel is used.
When training a convolutional layer of a neural network, it is exactly the kernel values that optimized.

\section{Backpropagation}
The back propagation algorithm is a fundamental part of how we train a neural network.
As this report does not focus on the details of how deep learning works, we will only discuss the
basic structure of the algorithm and not the specific mathemathical details.

The purpose of the backpropagation algorithm is to find the gradient of the model parameters with respect
to some output of the model - most commonly the loss function.
That is, if some output of the model $z$ is produced, the backpropagation algorithm finds the gradient over
the internal weights of the neural network.

Consider a neural network consisting of $L$ layers and lett $\vecsym{w}^L$ be the weights in the $L$'th layer.
Then the backpropogation algorithm will find 
$\frac{\partial z}{\partial \vecsym{w}^L}$ for each value of $L$.
For the purpose of model training where $z$ indicates the loss, updating each layer as 
$\vecsym{w}^L \leftarrow \vecsym{w}^L - \eta \frac{\partial z}{\partial \vecsym{w}^L}$
is called gradient descent.
Note the $\eta$ parameter that is used to update the weights.
This is called the learning rate, since it determines how fast the weights are updated during training.

The obvious problem is then how $\frac{\partial z}{\partial \vecsym{w}^L}$ is calculated.
The specifics of how that works can be seen on page 9 of Introduction to Convolutional Neural Networks \cite{tensorIntroduction}.



\section{Saliency maps}\label{sec:saliency_maps}
When training deep neural networks on image data,
it is often difficult to know exactly what the model is doing.
Especially in fields like medical image analysis, 
where the classification of a model might influence if a patient is correctly diagnoed or not,
it is very desireable to know what the model is doing to be able to critique the model.

In image analysis, a tool to try to understand models are the so-called \textit{saliency maps}.
A saliency map is supposed to be a visual representation what parts of the image are important to the model,
when it made a given prediction.

For instance, in a model supposed to classify if a picture of a bone is broken or not,
we want the model to look at the bone and not really anything else in the image.

Many methods exist to produce saliency maps, each with their own strengths and weaknesses.
One of the most popular ones is the gradient based method (explained below).
This method is popular because it is computationally efficient and the implementation comes almost for free if Backpropagation is implemented.

\subsection{Gradient based saliency maps} \label{sec:gradiant_saliency_maps}
As described in Section \ref{sec:partial_derivatives_of_scalar_over_vector},
if a function $f: \mathbb{R}^n \rightarrow \mathbb{R}$ is defined,
then the gradient of $f$ in a point $x\in\mathbb{R}$ is a vector from $\mathbb{R}^n$.
A special case of this, is where $f$ is an image classification model, that takes
an image (images can be vectorized, hence still a vector) and returns a vector of probabilities for
diferent classes.
Then the gradient in any probabilty of the output vector, will be of the same size as the input vector.
Since the input vector was an image, the gradient can also be interpreted as an image.
It is this property that is utilized when calculating gradient based saliency maps.

Often, these gradients are used as heatmaps over the image, to hightlight regions that 
supposedly contributed a lot to the classification decision of the model. 
For instance in the book Interpretable Machine Learning (chapter 10.2)\cite{interpretable-machine-learning}, the gradient based 
saliency map method is described to
''assign each pixel a value that can be interpreted as the relevance of the pixel to the prediction or classification of that image''.

One should however be carefull with that intepretation, as the gradient is just poiting to/away from the classification boundary.
Recent research is pointing toward problems with the usage of saliency maps as explained in the quote before\cite{false-hope}.
This report will not go deeper into exactly what information that can be extracted from a gradient based saliency map,
but just note to be carefull when using them.

\section{The RESNET architecture}
The RESNET architecture is a popular architecture for deep neural networks.
It was presented in 2015 and presented some promising results on the ImageNet dataset\cite{RESNET-paper}.
It is implemented PyTorch and other similar libraries, making it easy to implement.

\section{Model metrics}
To compare the performance of different models, different kinds of metrics can be used.
In the litterature on HAM10000, some researchers choose to look at all the classes presented 
in Table \ref{table:ham10000}.
Others choose to only look at weather the lesion is benign or malignant 
(see the \textit{severity} column in Table \ref{table:ham10000} - the \textit{pre-malignant} will be considered malignant).
To enable comparison with as many other studies as possible, we will report different metrics
on both problems as desribed in the following.
\subsection{Multiclass accuracy}
The most simple metric used here, is the accuracy, which is the percentage predictions that are correct.
\[
    \text{Accuracy} = \frac{\text{correct classifications}}{\text{total predictions}}
\]
Accuracy as a metric is not that good for problems with big class imbalance,
such as this one where a single class accounts for more than half of the data (see Table \ref{table:ham10000}).

\subsection{Binary accuracy}
The binary accuracy is defined exactly as the multiclass, but except of using all $7$ classes,
the considered classes are just benign and malignant.

\subsection{Malignant recall}
The general definition of recall in a binary classification problem is the percentage of the positive class
that is correctly classified.
The recall is defined as
\[
    \text{Recall} = \frac{\text{TP}}{\text{TP} + \text{FN}}
\]
Here $TP$ is the number of true positives, and $FN$ is the number of false negatives.
When refering to \textit{malignant recall}, we will think of the recall metric in the problem
where the considered classes are just benign and malignant.
As the name suggest, the positive class is malignant.

\subsection{Malignant F1 score}
% TODO: Write the F1 score definition

\subsection{Multiclass F1 score}
% TODO: Write the F1 score definition





\section{Statistical tests with conditioned variables}\label{sec:statistical_tests_with_conditioned_variables}
% TODO: Write about statistical tests with conditioned variables
% http://users.stat.umn.edu/~helwig/notes/ChiSquareTests_slides.pdf
% there are good examples with white/black death penalty which seem to be 
% the standard example