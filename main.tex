%--------------------
% Packages
% -------------------
\documentclass[11pt,a4paper]{article}
%\usepackage{gentium}

\usepackage[
backend=biber,
style=alphabetic,
sorting=ynt
]{biblatex}
\addbibresource{references.bib}

\usepackage[pdftex]{graphicx} % Required for including pictures
\usepackage[english]{babel} % Swedish translations
\usepackage[pdftex,linkcolor=black,pdfborder={0 0 0}]{hyperref} % Format links for pdf
\usepackage{calc} % To reset the counter in the document after title page
\usepackage{enumitem} % Includes lists

\frenchspacing % No double spacing between sentences
\linespread{1.2} % Set linespace
\usepackage[a4paper, lmargin=0.1666\paperwidth, rmargin=0.1666\paperwidth, tmargin=0.1111\paperheight, bmargin=0.1111\paperheight]{geometry} %margins
%\usepackage{parskip}

\usepackage[all]{nowidow} % Tries to remove widows
\usepackage[protrusion=true,expansion=true]{microtype} % Improves typography, load after fontpackage is selected

\usepackage{lipsum} % Used for inserting dummy 'Lorem ipsum' text into the template

\title{Understanding and checking for bias in medical image clasification models}
\author{Mikkel Bjørn Goldschmidt}

%-----------------------
% Set pdf information and add title, fill in the fields
%-----------------------
\hypersetup{ 	
pdfsubject = {},
pdftitle = {Understanding medical image clasification models},
pdfauthor = {Mikkel Bjørn Goldschmidt}
}

%-----------------------
% Begin document
%-----------------------
\begin{document}
\maketitle

\section{Introduction}
Image analysis can be used in the diagnosis of diseases in different ways.
A lot of the data used for diagnosing come in the form of images.
Examples of these are MRI, CT, PET, ultrasound but also just plain images of body parts.
Understanding these images is often done using machine learning models in the form of
some convolutinal neural network.
In their nature, these models are not easy to understand and can therefore end up biased without
the programmer knowing it.

This project focuses on the HAM10000 dataset\cite{Tschandl_2018}, which contains images of skin lesions.
Doctors diagnose these lesions into different classes, some more dangerous than others.
The challenge on the HAM10000 dataset is to find a model that can predict the class of a given image.

\section{The HAM10000 dataset}
The HAM10000 dataset contains 10000 images of skin lesions.
They are divided into seven classes as seen in the Table \ref{table:ham10000}.
A \textit{severity} has also been assigned each class.
These indicate if the lesion class benign, malignant or pre-malignant.
Malignant tumors are those where the cells are replicationg uncontrollably,
and there is a risk of these cells spreading to the rest of the body.
Beningn tumors can still hurt the patient, but are much less of a worry,
as they are not replicationg uncontrollably and hence do not lead to cancer.

\begin{table}[ht]
\begin{center}
\begin{tabular}{|c|c|c|c|c|c|c|}
\hline
Label    & Name                          & Severity      & Dataset prevalence \\ \hline
akiec    & Bowens disease                & pre-malignant & $3.27\%$           \\ \hline
nv       & Melanocytic nevi              & benign        & $66.94\%$          \\ \hline
mel      & Melanoma                      & malignant     & $11.11\%$          \\ \hline
bkl      & Benign keratosis-like lesions & benign        & $10.97\%$          \\ \hline
bcc      & Basal cell carcinoma          & malignant     & $5.13\%$           \\ \hline
vasc     & Vascular lesions              & benign        & $1.41\%$           \\ \hline
df       & Dermatofibroma                & benign        & $1.14\%$           \\ \hline
\end{tabular}
\end{center}

\caption{Overview of the HAM10000 dataset.}
\label{table:ham10000}
\end{table}

\printbibliography[title={Litterature}]

\end{document}